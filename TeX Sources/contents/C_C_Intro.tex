
\chapter{Introduction}
\label{chap:Introduction}

Since the beginning of peer to peer games, the game industry emerged towards online games in order to serve a growing number of simultaneous players \cite[192]{Nagygyorgy.2013}.
After general networking moved from small local token ring- and Ethernet-based networks \cite[1]{Smythe.1999}
towards huge online communities, gaming became a \textit{massive multiplayer online}-phenomenon (\citet[1]{Williams.2008}; \citet[1]{Wang.2012})
and publishers needed to offer servers to coordinate the high amount of players \cite[41]{Lee.2008} and to prevent cheating.
Nevertheless, all peers need to trust the publisher's server, whilst the server has to carry the costs of availability and network computation.
Consequently, power consumption, computation workload and likewise attached costs for \textit{networking and peering} are mainly carried by 'the central participant' of the network.
But sometimes the central element - e.g. a publisher - is not able to run the central server without the need to monetize its offerings.
One consequence was the rise of private dedicated servers, hosted by private persons (e.g. using a peer-to-peer architecture \cite[41]{Lee.2008}), which were 'detached from the publisher's direct influence'.
Additionally, the technical/mental hurdle to set up a dedicated server, whilst other more convenient games exist,
as well as the single point of failure 'out of the publisher's reach'
are considered reasons against a rise and large distribution of private, peer-owned dedicated servers.
Nowadays, along with reduced server costs\footnote{\hspace{0.1cm}Reduction of running costs due to Moore's Law (\cite{Mann.2000}, 2000). Here: less power consumption or increased sessions per hardware unit.},
many game publishers rely on in-game purchases, advertisements and data mining to target those players who are willing to pay more (\cite{Davis.2012}).
Consequently, consumers "[...] are regularly exposed to predatory behaviour from the games publishers, with microtransactions and loot boxes [...]" \cite[14]{Laneve.2019}.
Other distributed server architecture as proposed in the documents "A Distributed Server Architecture for Massively Multiplayer Online Games" from \citet{Khan.2006} and
"Matrix: adaptive middleware for distributed multiplayer games" from \citet{Besancon.2019}, to the writer's knowledge, never settled in the consumer world. \\
On the contrary, \gls{BCT} offers distributed computing in 'trustless environments' and
offers many of those by the (game) publishers needed characteristics.
Generally speaking, a \gls{BC} "[...] is an innovative technology, which can have a high impact in numerous industries,
such as healthcare [1], supply-chain [2], finance [3] and video games [4]. BC are append-only ledgers shared across a network of clients" \cite[84]{Besancon.2019}. \\
This thesis shall bring \gls{BCT} closer to practical use cases in the gaming industry.
Therefore, this study may especially be of interest to the following readers:
\begin{enumerate}
	\item Researchers interested in enhancing \gls{BCT} and affiliated mechanisms.
	\item Practitioners willing to understand how blockchain may affect the software (gaming) industry.
	\item Managers and administrative staff from the software (gaming) industry looking for potential savings of server operation costs.
\end{enumerate}
The remainder of the document is structured as follows:\\
This section will provide the problem statement, research design as well as guiding research questions in detail.
In section 2, a review of \gls{BCT} and included mechanisms to achieve consensus are presented.
Section 3 outlines recent applications of \gls{BCT} in the gaming industry as well as mechanisms to map known tabletop game mechanisms into digital contracts.
In section 4, a novel \gls{CM} for \gls{BCT} is presented and discussed in detail.
Finally, in section 5 the work is concluded and directions are provided for future research.



\FloatBarrier

\section{Current Situation}
\label{sec:CurrentSituation}
First of all, data and scientific research targeting the gaming industry is scarce.
Still it is known that  most "[...] game projects fail to meet their financial expectations [...]" \cite[17]{Bethke.2003}.
Nevertheless, the number of games is growing constantly (\cite{statista.com.2021}) and providers are forced
to minimize their (running) costs (\cite{Koster.2018}).
"Furthermore, small developers are finding it hard to stand out in a market dominated by giant companies that dwarf them through marketing or polish,
with the average indie developer earning only around \$10 thousand a year" (\citet[14]{Laneve.2019} using data from \citet[6]{Gamesetwatch.com.2014}. \\
Although there are several ways to cut costs, one of them is the use of dedicated servers which are paid by single peers or clans of the game network (\cite{Wikipedia.2021c}).
Especially in the long run, overall costs \cite[13]{Weilbacher.2012} may be reduced by these servers and help publishers to keep the game alive and establish a network of players and enthusiasts.
Some publishers offer dedicated servers (\cite{VALVe.2021}) as one possible solution for externalizing expensive hardware and maintenance.
Nevertheless, there are some reasons against dedicated gaming servers:
\begin{enumerate}
	\item \textbf{Giving away the source code} \\
	Behind this phrase, the publisher fears its game's code to be analyzed and cloned to provide any type of copycat product \cite[2]{Li.2014}.
	Two prominent examples for this phenomenon in the game industry is the massive imitation of the mobile game 'flappy bird' (\cite{Li.2014})
	as well as the co-evolution of the 'battle royale mode' (\cite{KooistraJ..2018}).
	Hence, these game mechanic ideas skyrocketed and (many) late adopters followed, but not because the code was given 'out of hands' in the first place.
	Therefore the argument can be considered obsolete as each game's general mechanics can be obtained by simply playing.
	Nevertheless, the fear is not negligible in general as sometimes source code is stolen (\cite{vice.com.2021}).
	
	\item \textbf{Update laziness} \\
	Games progress, therefore both 'security'- as well as 'game-mechanics'-updates
	shall comply with publishers' and gamers' expectations.
	But dedicated server providers are detached from the publishers' direct influence.
	The providers act by intrinsic motivation, ranging from \textit{control} over \textit{fame} up to \textit{monetization}.
	Hence, publishers as principals and providers as agents follow different incentives - which presents a \textit{Principal-agent problem} as generally described by \citet{STIGLITZ.1989}.
	
	\item \textbf{Nerdy peers} \\
	As a server needs to work 24/7 to serve whenever needed, there need to be maintainers \cite[211]{Lowell.2004}.
	These maintainers are not supposed to be casual gamers, but peers which offer expertise with web-servers, scripting and so forth (\cite{teamfortress.com.2021}).
	On the one hand, without these type of persons, the offer of dedicated servers is worthless and
	consequently cheap 'scalability and accessibility' can not be achieved.
	On the other hand, a game publisher has to consider an additional group of (silent) stakeholders.
	
	\item \textbf{Shadow server} \\
	A \textit{shadow server} is considered a server which runs on one players device, without the player's knowledge/consent.
	This is not recommended as players could 'force quit' the game or shut down their system (unintended).
	Consequently the (dedicated) shadow server would end to serve and remaining players are kicked out of their games which presumably leads to a bad consumer experience.
	
	\item \textbf{Lost peer management \& data collection} \\
	In the assumption that dedicated servers are generally detached from the publisher's reach,
	metrics and other data get lost although gamers stay within the game's ecosystem.
	Still, publishers have a "Need for Metrics" \cite[152]{Palmer.2002} to find flaws in their games.
	Thus, dedicated servers are seen as a misleading approach.
	
	\item \textbf{Performance} \\
	Bad performance of dedicated servers leads to bad consumer experience \cite[10]{Weilbacher.2012}.
	Nevertheless, the reasons for bad performance are manifold, from slow server devices not suitable to host their
	(large scale) games up to slow broadband connection \cite[10]{Weilbacher.2012}.
	
	\item \textbf{Integrity \& Cheating} \\
	The combination of nerdy peers and the possibility to modify game content (e.g. within a 'modding community')
	could lead to cheating behavior and consequent changed game characteristics (\cite{Morris.2003}).
	Although there are basic possibilities to check the integrity of server instances (\cite{Agosta.2003}; \cite{Deswarte.2004}),
	the publishers or other players can't rely on the system to be somehow silently manipulated.
	
\end{enumerate}
Still, the number of persons 'willing and able' to provide dedicated servers is seen low compared to the number of gamers playing in the manifold games.
To the writers knowledge, despite the enormous amount of mobile (Android) apps offering server-based network multiplayer
modes (\cite{itch.io.2021}),
there are only some mobile games, such as \textit{'Among Us'} and \textit{'Minecraft'} offering dedicated servers.
Those games have been designed for desktop primarily and have only thereafter been ported to mobile operating systems (Here: Android \& iOS).
Compared to the 'Hard-core' gamers, mobile and 'casual' gamers can be considered less likely to set up additional systems
to boost their game sessions due to reduced general involvement \cite[388]{Prugsamatz.2010}.
As small, independent developers might not be able to afford additional running servers,
there is a (growing) demand for distributed computing in the mobile gaming ecosystem.
At this point the interest in the present work originates from the insights into game development and knowledge about \gls{BCT}.
During the last years, \gls{BCT} emerged and offers the possibility to establish distributed networks to share data without a central authority.
For now, "[...] an analysis of the different projects, [...]" has shown "[...] that the main problems blockchain games face are related to scalability and transaction speed" \cite[47]{Laneve.2019}.
Nevertheless, if there is a possibility to tweak a Proof-/Voting-Based Consensus Model \cite[3]{Khan.2020} to fit the need of the gaming industry,
it might keep cooperative niche games alive or kick start some community driven game development projects \cite[14]{Laneve.2019}.



\FloatBarrier

\section{Problem Statement}
\label{sec:ProblemStatement}

The biggest issue seen for game publishers is the shortage of money, both during production and operation (\cite{Koster.2018}).
Sometimes, when games do not settle in the market, publisher servers need to be shut down (\cite{rockpapershotgun.com.2021}).
This beholds true for all genres as network- and cross selling effects play a key role for the survival of online (multiplayer) games \cite[45]{Rong.2018}. \\
Once game servers are shut down, players can only play offline, if at all.
Additionally, games are established in the perception of long term revenue.
Hence, once an investment decision for a game idea is made, long availability is a key feature.
Only an accessible game can continue to yield revenue.
Last, due to tight budget plans during a game's creation process, additional effort for dedicated servers is mostly not met as an active community has not yet existed.
Consequently, many game publishers are in a problematic situation between \textit{bankruptcy} and \textit{squeezing their consumers} for income generation \cite{Sotamaa.2021}.
Once a game is established, besides fixing bugs and implementing new features, running servers are a key driver of 'keep alive costs'.
The question arises whether there is any backend technology to further minimize some of the games running server costs.
The solution has to be:
\begin{enumerate}
	\item \textbf{Distributed}, to relieve the central running server from some/many workloads.
	\item \textbf{Trusted}, preventing malicious behavior and cheating.
	\item \textbf{Cross-plattform} applicable to take advantage of gamers using different operating systems.
	\item \textbf{Scalable \& Fast}, at least for some designated game niche use cases.
	\item \textbf{Leightweight} in terms of preventing heavy computation on single devices.
\end{enumerate}
On the first look, \gls{BCT} offers some of these characteristics.
Nevertheless, as "[...] it stands, blockchain technology does not seem applicable for the design of the most popular game genres such as first-person shooters or real-time strategy[...]" \cite[3]{Serada.2020}.
Still, "[...] several attempts have been made in this direction (e.g., EOS Knights, HyperDragons, and Epic Dragons), and many more are likely to follow [...]" \cite[3]{Serada.2020}. \\
Therefore this document will examine \gls{BCT} thoroughly to find a suitable application area.



\FloatBarrier

\section{Research Design \& Questions}
\label{sec:ResearchDesignQuestions}

The main research question which shall be answered with this thesis is:
\begin{center}
	\textit{"Can \gls{BCT} be used to reduce publisher's server costs whilst \\
		providing (mobile) players a suitable gaming experience?"}
\end{center}
To guide the further procedure, no explicit framework is used.
As a starting point the recent literature has been reviewed to reflect the state of research about \gls{BCT} and related \gls{CM}s.
The results of the grounding research and the \gls{CM}s performance characteristics are given in the chapter \hyperref[chap:BCT]{Blockchain Technology}.
Additionally, as \gls{BCT} is still pretty young regarding the field of information systems - starting from \cite{Nakamoto.2009} in 2009 -
there is only scarce literature to be found in the intersection with the gaming industry, yet.
Still, some applications and use cases of \gls{BCT} in games could be found.
"The industry has started its exploration on this topic by integrating traditional games with blockchain systems" \cite[1]{Min.2019b}.
Further more, \citet[1]{Min.2019b} state that blockchain games have already become an important component of decentralized applications and have held a considerable market capitalization.
Nevertheless, these applications are barely suitable for game play related issues.\\
As \gls{BCT} builds on trust and agreed rules, the mechanism called \gls{SC} has to be mentioned.
\gls{SC}s are rulesets which can be used to establish agreements between participants of \gls{BCT} networks.
As there seem to be reasons, which prevent even slow paced games to be conducted on \gls{BCT}, the following guiding question has been raised:
\begin{center}
	\textit{"Which kind of \gls{SC}s need to be established to cover typical in-game mechanics?"}
\end{center}
Both, existing games using \gls{BCT} as well as game specific \gls{SC}s are given in chapter \hyperref[chap:BlockchainInGames]{Blockchain in Games}.
Moreover, during the search for suitable \gls{CM}s for games hosted on \gls{BCT},
it appeared that games with realtime-mechanics, such as first person shooters and \gls{MMORPG},
can be considered as not suitable for \gls{BCT} backbones \cite[19]{Serada.2020}.
Therefore, fast paced games became out of scope as distributed computation lacks performance regarding speed most of the time \cite[3]{Serada.2020}.
Thus this present work concentrates on slow paced games, such as asynchronous 'turn based (strategy)'-games \cite[1]{Bergsma.2008} and raises a second guiding question:
\begin{center}
	\textit{"What is the best fitting \gls{BCT} \gls{CM} to cover an asynchronous game play scenario?"}
\end{center}

To address this research gap, the different Proof-/Voting-Based Consensus Models are evaluated and
a novel \gls{CM} (Chapter: \hyperref[chap:PoT]{Proof-of-Turn}) is proposed.
\gls{PoT} is a special merge of existing protocols to fit the need of turn based games and
remains a theoretical proof of concept for a (new) Proof-/Voting-Based \gls{CM}.
Consequently the research hypothesis is:
\begin{center}
	\textit{"The \gls{PoT} \gls{CM} leads to a reduction of server running costs for game publishers."}
\end{center}
Additionally the following working assumptions are used:
\begin{enumerate}
	\item \textit{Throughout a game a majority of peers is constantly online.}
	\item \textit{The network offers enough transmission performance (throughput/bandwidth).}
	\item \textit{Every peer has the needed storage available.}
	\item \textit{Depending on the 'real world'-scenario, the publisher's servers help on special \\
		occasions (e.g. a trusted timestamp).}
	\item \textit{Symmetric and asymmetric encryption is secure.}
\end{enumerate}
Due to the intersection of this research between \textit{business administration} and \textit{computer science},
the document falls into the research area of \textit{economics of information systems}. \\
Last, in other contexts gamers, players, peers and nodes are sometimes used as synonyms,
but in the following they need to be distinguished:
\begin{enumerate}
	\item \textbf{Gamers} are persons, who play a game.
	\item \textbf{Players}, which are considered digital in-game characters, are controlled by \textit{gamers}.
	\item \textbf{Nodes} are hardware systems, which is part of a \gls {BCT}-(gaming-)network.
	\item \textbf{Peers} are persons, who can be \textit{gamers} but predominantly own or maintain a node.
\end{enumerate}
Upcoming the core principles of \gls{BCT} are described.



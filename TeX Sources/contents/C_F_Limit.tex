
\chapter{Conclusion \& Outlook}
\label{chap:ConclusionAndOutLook}

To answer the primary research question ..
\begin{center}
	\textit{"Can \gls{BCT} be used to reduce publisher's server costs whilst \\
		providing (mobile) players a suitable gaming experience?"}
\end{center}
.. the two leading questions have been discussed and answered within chapter 
\hyperref[chap:BCT]{Blockchain Technology} and chapter \hyperref[chap:PoT]{Proof-of-Turn}.
Nevertheless, the primary research question cannot be answered within a straight boolean value spectrum.

\noindent From a purely algorithmic perspective, there is a solution to most challenges (e.g. \hyperref[sec:PileOfCards]{Card draw}).
But that there is a viable algorithm is only part of the answer.
This thesis brings to mind that every additional required algorithm is a trade-off:
What is gained in pure feasibility on one side, is payed in additional development effort, higher complexity (and lower maintainability), lower system performance (and higher transaction costs).
The risk of faulty technicality decreases, but other risks increase in return.
In the case of games, dropping out players due to limited chances to win would be only one of the problematic cases.
Especially the higher complexity should be evaluated throughout, before the need of cumbersome fixes arise.

\noindent Therefore, whilst \gls{CM}s may function unexpectedly good in a theoretical document,
the final '\textit{design of upper level (gaming) software}', '\textit{additional programming effort}' as well as
'\textit{cooperation and coordination of gamers}' (e.g. players do only regularly quit game sessions)
are major factors which have to be considered to answer the research question throughout.

\noindent This additional research would offer further insights to adjust and verify first the given working assumptions
and second the \hyperref[script:GraphPlots]{plots} containing (only) sample data.
Hence, sample games have to be invented, implemented and analyzed throughout.
Unluckily, the concomitant workload would have exceeded the scope of this document by far.

\noindent Nevertheless, the tight budgets in the gaming industry call for mechanisms to improve
their games backend and \gls{PoT} offers a solution for slow paced games promising low (networking) costs.
Thus, finding sample applications conducting \gls{PoT} is assumed not to be too far fetched,
as described in the \hyperref[sec:Outlook]{Outlook}.

\pagebreak



\section{Outlook}
\label{sec:Outlook}

The outlook focuses on the \gls{PoT} \gls{CM}, but abstracts from the use case of games.
Hence, research in other possible application's use cases with \gls{PoT}
and '\textit{Adaption in the Wild}' is aimed for:
\begin{enumerate}
	\item \textbf{A truly distributed chat} \\
	Given an initial interconnection, every pair of devices could establish a \gls{BC} for their chat protocol.
	To the writer's knowledge, every recent messenger relies on central server technology.
	Still the most messengers lack funding.
	If users had an especially high need for privacy wherein only their devices had encryption keys, 
	they could use a messenger build upon \gls{BCT} using a lightweight \gls{CM} as \gls{PoT}.
	Interconnection could still suffer speed as the other node peer might be offline,
	but speed up turns, down to $\sim$ten seconds seem possible.
	During inactivity, dynamic turn time may be used as well.
	Here, if turn time was set to long (sleeping chat),
	a direct message to the \gls{LN} could trigger its turn's end and decrease transaction delivery times.
	Last, it would be reasonable to sacrifice instantaneous delivery for an increased level of security,
	as non existing central servers can not loose their integrity.
	
	\item \textbf{Cryptocurrencies and parliamentary elections} \\
	The ability to rewrite the chain, grounded on any network vote,
	makes \gls{PoT} futile both for cryptocurrencies and parliamentary elections.
	Still, e.g. the possibility of flooding attacks and the setting in \textit{permissioned networks}
	limits the practicality in this regard.
	Therefore, it is discouraged to use \gls{PoT} in such critical infrastructures in general.
	
	\item \textbf{The citizens trust} \\
	Awarding of public administration contracts is mostly conducted using a public administration's central server.
	In this scenario, each company's offering is seen mutually exclusive to the others and all together create a race condition.
	Moreover, knowing the bids of other companies creates unfair advantage to the informed
	party - especially if the informed party still has to claim its price tag.
	Not only does the \textit{offset revelation} mechanism (independent from \gls{BCT})
	promise a fair and transparent match making in the market.
	\gls{PoT} offers here a lightweight (computation) and balanced (writing permission) distributed \gls{CM}.
	Hence, \gls{PoT} may help to lighten opaque administrative structures and raise the stakeholders' trust.
	
	\item \textbf{Responsible disclosure} \\
	Just found vulnerabilities in software systems, formally known as zero-day vulnerabilities
	as well as startling information gained from whistle-blowing activity endanger not only the targeted institution(s),
	but also the investigative journalist(s) and security researcher(s).
	The publishing party (person) is not dependent on any server to stay online or
	has to fear that the central server is taken down from any higher authority.
	Still \textit{offset revelation} can be used to inform affected parties first,
	formally known as 'responsible disclosure'.
	If implemented accordingly, the publishing party may stay anonymous and
	gains additionally the ability to publish the encrypted data from a different node than the affiliated key(s).
	Here, if nodes stay offline the \gls{PoT} algorithm may be adjusted to choose the next node according to online activity.
	Inactive nodes are - after a network vote - skipped.
	Hence, the publishing party is enabled to stay offline, until the right time to publish has come.
	This scenario, again, shows the adaptability of \gls{PoT}.
	Last, as \gls{PoT} assumes that nodes write their date themselves and
	nodes do not compete for tokens gained from mining, the possibility is
	high for the publishing party to release the data without intermediaries.
	Additionally, as some critical acts are sometimes discovered, not whilst conducting, but during the phaseout,
	transferring cryptocurrency tokens into traditional currencies was a risk for the publishing party.
	Therefore, the absence of cryptocurrency tokens in \gls{PoT} protects publishing parties by design.
			
\end{enumerate}

\noindent Next to \gls{PoT}, this thesis gives some hints on future research fields like
'finding an optimal child-chain size' for different use cases. \\
All these and other scenarios call for further research and '\textit{Adaption in the Wild}'.




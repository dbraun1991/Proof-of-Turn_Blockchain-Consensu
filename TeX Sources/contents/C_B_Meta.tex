

\addcontentsline{toc}{section}{Abstract}
\section*{Abstract}
\label{sec:Abstract}

\textbf{TL;DR}: blah \\

This master thesis deals with Blockchain Technology in turn based online peer to peer games. \\

This thesis argues(/shows) that a distributed database, based on the Blockchain technology, is capable to enable game publishers to cut costs in the means of running servers, whilst raising the trust level among the players in tunr based games. \\



\addcontentsline{toc}{section}{Preamble}
\section*{Preamble}
\label{sec:Preamble}
The following definitions are probably known by the reader, nevertheless they are given beforehand for a better reading experience and an improved understanding:
\begin{enumerate}
	\item \textbf{Blockchain Technology} \label{def:BCT} \\
	\textit{Blockchain Technology} (\textbf{BCT}) ".. is a form of distributed ledger technology, deployed on a peer-to-peer network where all data are replicated, shared, and synchronously spread across multiple peers. The technology allows actors participating in the network to perform, sign, and announce transactions by employing public key cryptography. Transactions are executed following a consensus protocol operated by specific nodes to ensure the validity of transactions requested by other peers in the network, and to synchronize all shared copies of the distributed ledger. During a consensus protocol execution, the data of valid transactions, along with other required metadata concerning the network, and the hash of the previous block are bundled into a block using hashing functions. The essential and key property reflecting \hyperref[def:BCT]{BCT} architectures is that each block contains the hash of their predecessor, therefore linking all prior transactions to newly appended transactions; the blocks therefore form a chain with the aim of establishing a tamper-proof historical record"
	\cite[13]{Butijn.2020} \\
	Further details can be found in the chapter \hyperref[sec:BlockchainTech]{Blockchain Technology}.
	
	\item \textbf{Cheating}	\label{def_Cheating} \\
	Illegal behavior, which leads to an advantage for the cheating person.
	Although the rules of the game shall prevent such behavior, the cheating person found a way to circumvent the game mechanics.
	
	\item \textbf{Cryptocurrency} \label{def_Cryptocurrency} \\
	A cryptocurrency is a construct of hypothetical value - just  as any other currency - but it is not governed by a central institution but rather by a distributed algorithm.
	The transactions are secured by cryptographic functions as well as mathematical problems.
	
	\item \textbf{Dark Patterns} \label{def_DarkPatterns} \\
	Mechanics withing games which nudge players, more or less obtrusive, to behave in a specific way.
	If a player gives in to the intrusiveness, a (hidden) negative or neutral consequence follows.
	\hyperref[def_DarkPatterns]{Dark patterns} are seen as malicious mechanics by the publishers.
	
	\item \textbf{Gamers, Players, Peers and Nodes} \label{def_GamersPlayersPeersAndNodes} \\
	Although in other context gamers, players, peers and nodes could be seen as synonymous, here they need to be distinguished.
	A node will be considered as a hardware system, which is part of a (gaming-)network.
	Players are in game character, who are controlled by gamers.
	Gamers are persons, who e.g. own nodes.
	Finally, peers are persons, who can be gamers but predominantly own or maintain a node.
	
	\item \textbf{Genesis Block} \label{def_GenesisBlock} \\
	The \hyperref[def_GenesisBlock]{genesis block} ".. is the first block in the chain and has a predetermined hash" \cite[181]{Oliveira.19.02.201921.02.2019}.
	
	\item \textbf{Mining} \label{def_Mining} \\
	The process of solving a computational hard mathematical puzzle by finding the right nonce (a random number) for the block-header based on information of the prior block.
	The nodes executing the calculations are referred to as miners in the \hyperref[def_Cryptocurrency]{Cryptocurrencies} nomenclature.
	The first miner to finish creates the next block and is rewarded by receiving an amount of the underlying \hyperref[def_Cryptocurrency]{Cryptocurrency} (BUTIJN, p. 60, 2020). (???)
	
	\item \textbf{Principal-Agent Problem} \label{def_PrincipalAgentProblem} \\
	"The principal-agent literature is concerned with how one individual, the principal (say an employer), can design a compensation system (a contract) which motivates another individual, his agent (say the employee), to act in the principal's interests" (Stiglitz, p. 241, 1989).
	
	\item \textbf{Transactions and Smart Contracts} \label{def_TransactionsAndSC} \\
	Every entry in a Blockchain is considered a transaction.	
	If data in a transaction is constraint to a set of rules, the transaction is called a Smart Contract (SC).
	
	\item \textbf{Trolling} \label{def_Trolling} \\
	Trolling in online communities can either be cheating or malicious/hostile behavior. 
	
\end{enumerate}